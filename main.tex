\documentclass{book}
\usepackage{import}
\import{/home/vatican/rituale-institutionum/packages/}{packages}
\geometry{a5paper,hdivide={1.5cm,*,1.5cm},vdivide={1.5cm,*,1.5cm}}
\begin{document}
\pagestyle{empty}
\begin{center}
    RITO PARA INSTITUIÇÃO DO MANDATO DE MINISTRO EXTRAORDINÁRIO DA SAGRADA COMUNHÃO EUCARÍSTICA
    \vspace{.2cm} \\
    \textcolor{red}{(APÓS A HOMILIA)}
    \vspace{.2cm} \\
    APRESENTAÇÃO DOS CANDIDATOS
    \vspace{.2cm} \\
\end{center}
\begin{flushleft}
    \textbf{Formador (Apresentador):} Queiram aproximar-se, os que serão constituídos Ministros Extraordinários da Sagrada Comunhão Eucarística: \textcolor{red}{N. N}. (Cada um diz PRESENTE, do lugar onde se encontra, e se coloca diante do Altar).
    \vspace{.1cm} \\
    \textbf{Formador (Apresentador):} A Igreja de Deus que está na Arquidiocese de Olinda e Recife, territorialmente localizada no VICARIATO PAULISTA, sob a responsabilidade do Vigário Episcopal Monsenhor Osvaldo Lopes, solicita a Vossa Revma se digne constituir Ministros Extraordinários da Sagrada Comunhão Eucarística e das Exéquias, os cristãos aqui presentes e conferir-lhe a missão canônica na qual serão investidos.
    \vspace{.1cm} \\
    \textbf{Presidente:} Podeis dizer-me se os candidatos aqui presentes foram preparados para assumir as funções de Ministros Extraordinários da Sagrada Comunhão e das Exéquias? Podeis também garantir sua idoneidade?
    \vspace{.1cm} \\
    \textbf{Formador (Apresentador):} Sim, os candidatos aqui presentes foram escolhidos pelos seus respectivos Párocos e convenientemente preparados para assumir as funções do Ministério Extraordinário e também podemos afirmar que foram considerados dignos.
    \vspace{.1cm} \\
    \textcolor{red}{\small Estando todos da comunidade sentados, e somente os futuros Ministros em pé o Presidente da Celebração se dirige aos candidatos falando sobre: a) A importância dos seus serviços na Igreja e para a Igreja; b) A grandiosidade da Eucaristia e do Culto Dominical em suas vidas e para a vida da comunidade cristã; c) A obrigação de se distinguir pela fé, santidade de vida, espiritualidade, humildade e caridade fraterna e pelo testemunho cristão.}
    \vspace{.2cm} \\
\end{flushleft}
\begin{center}
    EXORTAÇÃO
    \vspace{.2cm} \\
\end{center}
\begin{flushleft}
    \textbf{Presidente:} Aos nossos irmãos, cujos nomes foram apresentados, é conferido o ofício pelo qual eles mesmos podem tomar a Santíssima Eucaristia e administrá-la aos outros, levá-la aos doentes e administrar o Viático.
    \vspace{.1cm} \\
    Vós, caríssimos irmãos e irmãs, que sois investidos de tão sublime ofício na igreja esforçai-vos no crescimento da vida cristã, pela fé e os bons costumes e a viver mais fervorosamente deste mistério da unidade e da caridade, pois, sendo muitos, formarmos um só corpo, nós que participamos de um só pão e de um só cálice.
    \vspace{.1cm} \\
    Assim, ao distribuirdes a Comunhão, praticareis com mais fervor a caridade, como o Senhor Jesus ordenou quando disse aos seus discípulos, ao dar-lhes a comer o seu corpo: ``O que vos mando é que vos ameis uns aos outros''; e ajoelhando-se diante deles, lavou-lhes os pés. Como o próprio Jesus explicou mais adiante: ``Se eu, que sou Mestre e Senhor, vos lavei os pés, também vós deveis lavar os pés uns dos outros. Eu vos dei o exemplo, para que façais o que fiz''(cf. Jo 13,14-15). Jesus em sua humildade ensina o serviço. Vós, caros Ministros, também sois chamados a servir com alegria e com disposição, tornando-vos uma presença viva dentro da sua comunidade. Tendo tudo isso presente vos pergunto:
    \vspace{.2cm} \\
\end{flushleft}
\begin{center}
    COMPROMISSO
    \vspace{.2cm} \\
\end{center}
\begin{flushleft}
    \textbf{Presidente:} Quereis assumir o mandato de Ministros Extraordinários da Sagrada Comunhão Eucarística e Exéquias para distribuir aos vossos irmãos e irmãs a Palavra de Deus e a Eucaristia, com o intuito de servir e edificar a Santa Igreja Católica?
    \vspace{.1cm} \\
    \textbf{Candidato:} Quero
    \vspace{.1cm} \\
    \textbf{Presidente:} Quereis dedicar-vos com o máximo cuidado e reverência na conservação e distribuição da Sagrada Comunhão Eucarística administrando-a com zelo e conformando vossas vidas ao sacrifício de Cristo?
    \vspace{.1cm} \\
    \textbf{Candidato:} Quero
    \vspace{.1cm} \\
    \textbf{Presidente:} Quereis servir à vossa comunidade dedicando-vos a celebrar com fé os mistérios da vida e da morte, seja levando a comunhão aos doentes, seja presidindo, quando necessário, as celebrações da Palavra e das Exéquias?
    \vspace{.1cm} \\
    \textbf{Candidato:} Quero
    \vspace{.2cm} \\
\end{flushleft}
\begin{center}
    BÊNÇÃO \\
    (Candidatos ajoelhados)
    \vspace{.2cm} \\
\end{center}
\begin{flushleft}
    \textbf{Presidente:} Ó Deus, fonte de toda luz e bondade, que enviaste vosso Filho Unigênito, Jesus Cristo, para revelar aos homens e mulheres o mistério do vosso amor, dignai-vos abençoar estes nossos irmãos e irmãs escolhidos para o Ministério Extraordinário da Sagrada Comunhão. Fazei que desempenhando com dignidade e dedicação esse serviço, distribuindo fielmente o pão da vida aos seus irmãos, sejam confortados pela virtude deste sacramento e, possam participar um dia para sempre do banquete celeste. Por Cristo Nosso Senhor.
    \vspace{.1cm} \\
    \textbf{Todos:} Amém.
    \vspace{.1cm} \\
    \textcolor{red}{(Em Seguida, com todos de pé, o Presidente abençoa as vestes litúrgicas que serão vestidas pelos candidatos.)}
    \vspace{.1cm} \\
    \textbf{Formador (Apresentador):} Agora, já investidos nas funções de Ministros Extraordinários da Sagrada Comunhão, convido os acompanhantes escolhidos a vestirem os novos ministros com a BATA, veste do ministro para as funções a ele delegadas.
    \vspace{.1cm} \\
    \textcolor{red}{(JÁ VESTIDOS, TODOS SE VOLTAM E SAÚDAM A ASSEMBLEIA SENDO ACOLHIDOS PELOS PRESENTES E ABRAÇADOS PELOS PÁROCOS)}
    \vspace{.1cm} \\
    Segue a celebração com a oração universal.
\end{flushleft}
\end{document}
